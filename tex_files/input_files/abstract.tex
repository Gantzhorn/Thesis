\noindent Recently, there has been an increased focus on drastic changes in the Earth's climate systems, known as tipping points. These events pose potentially unforeseen changes not only for mankind but potentially the entire biosphere, and while reasoning about a system after a tipping point has occurred is inherently difficult, advancements in statistical methods for stochastic differential equations \cite{SplittingSchemes} have shown great promise in predicting the tipping points themselves. Specifically, stochastic differential equations have been applied in the estimation of a key tipping point in the Atlantic Meridional Overturning Circulation \cite{Ditlevsen2023} using an assumption-lean model.  We continue this work, broadening the potential applications by presenting five additional models that share deterministic structure with the original model but use other diffusion terms. Additionally, we introduce a natural extension to the part of the system modelling the changes in its deterministic dynamics. For each of these models we implement efficient estimators in the \code{R}-language\cite{Rlang}. The estimators from these models are scrutinized in a simulation study, in which we consider the error, computation speed, robustness and numerical stability of our implementations. Moreover, we look into and contrast different approaches to practical implementations of estimation methods for stochastic differential equations more generally: We introduce an estimator, which almost exclusively rely on numerical methods and compare this estimator with less numerically founded methods. Eventually, two of the tipping models are applied to the same proxy for the strengh of the Atlantic Meridional Overturning Circulation. Finally, we discuss the findings from the simulation study and comment on various possible ways to do further research on this topic. Additionally, our results for the tipping of the Atlantic Meridional Overturning Circulation is compared to the original work \cite{Ditlevsen2023}; and we reflect on how these models can complement one another in general.