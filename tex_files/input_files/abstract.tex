\noindent Recently, there has been increased focus on drastic changes in the Earth's climate, known as tipping points. These often sudden events can have severe consequences not only for mankind but for the entire biosphere. While reasoning about a system after a tipping point has occurred is inherently difficult, advancements in statistical methods for non-linear stochastic differential equations \cite{SplittingSchemes} have shown promise in predicting a key tippng point in the the Atlantic Meridional Overturning Circulation \cite{Ditlevsen2023} using an assumption-lean model. Despite the extensive previous work, the model has only been analyzed with additive noise. We present five additional models that shares deterministic structure with the original model but uses other diffusion terms. Additionally, we introduce a natural extension to the part of the system modelling the changes in its deterministic dynamics. With the original additive model this totals 6 different ways to model systems with potential tipping points. The estimators from these models are scrutinized in a simulation study. In the study we consider the error, computation speed, robustness and numerical stability of our implementations of the estimators. Moreover, we look into and contrast different approaches to practical implementations; introducing an estimator, which almost exclusively rely on numerical methods. Eventually, two of the models are applied to the same proxy for the strengh of the Atlantic Meridional Overturning Circulation as examined in previous work \cite{Ditlevsen2023}. Finally, we discuss the findings from the simulation study and comment on various possible ways to do further research on this topic. Additionally, our results for the tipping of the Atlantic Meridional Overturning Circulation is compared to the original work and we reflect on how these models can complement one another in general. Finally, we discuss further extensions to the models.