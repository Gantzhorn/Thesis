\noindent Recently, there has been an increased focus on drastic changes in the Earth's climate systems, known as tipping points. These events pose potentially unforeseen changes not only for mankind but also the entire biosphere. For statisticians, modelling the consequences of these phenomena is a great challenge, because reasoning about systems after a tipping point has occured with observations only prior to it is inherently difficult. However, it is also within our interests to predict when a potential tipping point is reached in the first place. Advancements in statistical methods for stochastic differential equations \cite{SplittingSchemes} have shown great promise in exactly this. Specifically, stochastic differential equations have been applied in the estimation of a key tipping point in the Atlantic Meridional Overturning Circulation with an assumption-lean model \cite{Ditlevsen2023}. We continue this work, broadening the potential applications by presenting five additional models that share deterministic structure with the original model but use other diffusion terms. Additionally, we introduce a natural extension to the part of the system modelling the changes in its deterministic dynamics. For each model, we implement efficient estimators in the \code{R} language\cite{Rlang} that are scrutinized in a simulation study, where we consider the error, computation speed, robustness, and numerical stability of our implementations. Moreover, we compare different approaches to practical implementations of estimation methods for stochastic differential equations more generally: We introduce an estimator, which almost exclusively relies on numerical methods and compare this estimator with less numerically founded methods. Eventually, two of the tipping models are applied to the same proxy for the strength of the Atlantic Meridional Overturning Circulation. Finally, we discuss the findings from the simulation study and comment on various possible ways to do further research on this topic. Additionally, our results for the tipping of the Atlantic Meridional Overturning Circulation are compared to the original work \cite{Ditlevsen2023}; and we reflect on how these models can complement one another in general.