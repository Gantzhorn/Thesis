For the simulations we use the so-called Scalar-weak order 2.0 Itô-Taylor method \cite[algorithm 8.5]{Srkk2019}. Apart from this, detalis on how the simulations are carried out, can be found in section \ref{appendix:simMethods} in the appendix. The schemes that we derive for the respective types of noises are shown in equations (\ref{eq:OUSim} - \ref{eq:jacobiDiffusion}). A method for each scheme is implemented in \code{R}. The methods can be called with specific parameter choices, values for $\tau_c$ and $t_0$ as well as a step length, $\Delta t$ and starting point, $X_0$. If no starting point is specified the methods picks the fix point in the stationary part of the process as the starting point. Additionally, an argument is implemented such that we can opt for stopping the process at earlier time points than $\tau_c$.