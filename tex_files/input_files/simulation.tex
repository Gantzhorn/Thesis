As said in section \ref{subsubsec:Discretization} the simulations are done with the Scalar-weak order 2.0 Itô-Taylor method \cite[algorithm 8.5]{Srkk2019}. The concrete schemes that we derived for the respective types of noises are shown in equations (\ref{eq:OUSim} - \ref{eq:jacobiDiffusion}). A method for each scheme is implemented in \code{R}. All methods can be called with specific parameter choices, values for $\tau_c$ and $t_0$ as well as a temporal resolution, $\Delta t$ and starting point, $x_0$. Depending on the $\tau_c$, $t_0$, and $\Delta t$ the method dynamically picks the appropriate number of samples to draw for $\Delta W_k\sim\mathcal{N}\left(0,\Delta t\right)$. The methods allow for positive $\lambda_0$ and negative $A$, but recall that this only correspond to choosing between the positive and negative versions of (\ref{eq:standardStochasticForm}). If no starting point is specified the methods picks the fix point in the stationary part of the process as the starting point; note that this of course is done with the appropriate branch of (\ref{eq:fixedPoint}) depending on the version of (\ref{eq:standardStochasticForm}), i.e. the sign of $A$. Additionally, an argument is implemented such that we can opt for stopping the process at earlier time (or later) points in time than $\tau_c$. This is useful, when we want to evaluate the models' ability to predict the tipping point based on how far from it we have samples.
\subsection{Overview of the estimation methods}
\subsubsection{The stationary parts}

\subsubsection{The dynamic parts}

\subsection{Fitting the $\nu$-parameter}

\subsection{Early warning analysis}

\subsection{The numerical Strang}

\subsection{Model misspecification}