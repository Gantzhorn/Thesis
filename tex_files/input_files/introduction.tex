\subsection{Background and motivation}
Stochastic differential equations have long been a popular choice for modelling processes in physics, biology and a wide range of other fields of study. An often celebrated quality of these models is their innate ability to capture the stochasticity of the dynamics in these systems. However, another situation they are frequently used in, is when one works with a system so complex that it is either not entirely understood or when the complexity makes the model intractable. In these cases, stochastic differential equations allow us, often with great succes, to model the well understood- or simple parts as deterministic dynamics, whereas the unknown- or complex terms act enter as random. We work with one-dimensional stochastic differential equations commonly written on the following parametric form
\begin{align}
    \mathrm{d}X_t = b(X_t, t;\theta)\mathrm{d}t + \sigma\left(X_t, t; \theta\right)\mathrm{d}W_t,
\end{align}
where the solution, $X_t\in\mathbb{R}$, is called an Ito process. The stochasticity in the system comes from, $W_t$, which is a Wiener process. The function in the deterministic- and stochastic part are named drift- and diffusion terms, respectively, and the our overall goal is typically to do some sort of inference about the parameter, $\theta\in\Theta\subseteq\mathbb{R}^p$, that the functions depend on. 

An example of an application of stochastic differential equations is the modelling of bifurcation points, commonly known as tipping points for ecological systems. In the study of dynamical systems, a bifurcation point is a point in time, where the system's qualitative structure changes significantly. As such, predicting a tipping point typically involves considering systems that can exist in a multitude of somewhat stable states and then trying to predict when or what needs to be fulfilled for a \textit{transition} between those states to occur. In nature there are countless of these systemes such as: The landmass of Greenland that exists in a state with- or without an ice cap; or the Amazon existing in a state as a rainforest, - a savannah or even as a dessert.

Yet, the overall dynamics of such systems are typically intricate. Even with extensive domain knowledge one is often at a loss, when trying to predict complex phenomena such as tipping points. However, tipping comes in different types and knowing what type of tipping our system has the potential to undergo can vastly simplify the model for our system's dynamics, locally. An example of this is the model used to estimate the tipping point of the important climate system: The Atlantic Meridional Overturning Current (AMOC) \cite{Ditlevsen2023}. The AMOC is a key system of ocean currents and it plays a pivotal role in the Earth's climate; it exists in two states: "on" and "off" refering to whether there is a current. The specific type of bifurcation this system can undergo is known as a saddle-node bifurcation \cite{Ditlevsen2023}\cite{Strogatz2019_gv}. As other bifurcations the saddle-node bifurcation has a so-called normal form of its dynamics 
\begin{align}
    \mathrm{d}x_t = \pm\left(x_t^2 - \lambda\right)\mathrm{d}t. \label{eq:normalFormIntroduction}
\end{align}
A normal form is an approximate form of the system's dynamics. Systems that can undergo saddle-node bifucartions are well described by this normal form, when the bifurcation parameter, $\lambda$, is close to the bifurcation point, $\lambda_c$, and this is exactly what is exploited in the model. We assume that we are sufficiently close to the tipping point such that the system's dynamics are well described by the normal form, then we model the tipping point by the following stochatic differential equation
\begin{align}
    \mathrm{d}X_t &= \pm(A(X_t - m) + \lambda_t)\mathrm{d}t + \sigma\sqrt{\left(aX_t^2 + bX_t + c\right)}\mathrm{d}W_t, \label{eq:normalFormModelIntroduction}\\
    \lambda_t &= \lambda_0\left(1 - \max\left\{\frac{t - t_0}{\tau_c}\right\}\right)^\nu \label{eq:lambdaRampDefinitionIntroduction}
\end{align}
which is system very similar to the normal form; later we argue how one easily arrives at (\ref{eq:normalFormModelIntroduction}) from (\ref{eq:normalFormIntroduction}). The parameters we want to estimate are $\theta = \left(A\; m\; \sigma\; \lambda_0\; \nu\right)^\top$. The values $a, b$ and $c$ is chosen, when one picks a specific diffusion and such that the square-root is defined. The way the bifurcation parameter, $\lambda_t$, is modelled here implies that we have observations from the system in some stable state before time $t_0$, after which a \textit{ramping} of the parameter begins. We show that the tipping point occurs at $\lambda_t = 0$, corresponding to the point in time, $t_c = t_0 + \tau_c$.

The model we present here is a generalization of \cite[equation (1)]{Ditlevsen2023} in two ways: Firstly, the noise in the original work was assumed additive. We extend this to include diffusion terms from the Pearson diffusions: A class of stochastic differential equations on the form
\begin{align}
    \mathrm{d}X_t &= -\alpha_0\left(X_t - \mu_0\right)\mathrm{d}t + \sigma\sqrt{\left(aX_t^2 + bX_t + c\right)}\mathrm{d}W_t. \label{eq:pearsonDiffusionsIntroduction}
\end{align}
The Pearson diffusions include a process with additive noise - commonly known as the Ornstein-Uhlenbeck process. This process corresponds to picking $a = b = 0$ and $c = 1$ in (\ref{eq:pearsonDiffusionsIntroduction}). The Ornstein-Uhlenbeck process belongs to a special class of Pearson diffusion refered to as the \textit{ergodic} Pearson diffusion in the litterature. This is a class of model for which there exists efficient estimators. The ergodic Pearson diffusion also share many properties with the Ornstein-Uhlenbeck; and, as we will show, this makes using the diffusion term from each of these models a natural choice. In sum, we introduce five new versions of the normal form based model for estimation of tipping points bringing the total number available models up to six.

Secondly, the ramping in the original work \cite[equation (2)]{Ditlevsen2023} can be seen as a special case of (\ref{eq:lambdaRampDefinitionIntroduction}) with $\nu = 1$. We shall see that changing $\nu$ can alter the dynamics quite drastically. However, the extension is still natural in the way that it preserves the key property that the tipping time is given by $t_c = t_0 + \tau_c$.

Our work with these models focuses on and solves some of the challenges that come with both types of extensions. This includes the use of the Strang Splitting scheme\cite{SplittingSchemes} to approximate the transition density of (\ref{eq:normalFormModelIntroduction}) as well as the use of the rich framework for numerical optimization provided by the \code{R}-language along with our own implementations of sophisticated numerical methods \cite{Rlang}. In practical implementations, the general aim is, of course, to keep the methods as efficient and numerically stable as possible. Though, our work also focuses on the consequences of varying degrees of automation in the implementations. Specifically, we examine the effect that relying more on numerical methods over analytical results has on the quality of our estimator, the computational speed and numerical stability. To this end, we also consider how easy it is to use estimators based on these different degrees of automation in practial applications. Finally, with one of our newly developed models we investigate the estimated tipping time the AMOC from previous work and in doing so address the robustness of these estimates.
\subsection{Key contributions}
For a quick overview we list the key contributions of this thesis.
\begin{enumerate}
    \item \textbf{Model development and -extensions:} We introduce five models that takes the diffusion terms from the Pearson diffusions and use them to extend the existing methodology for prediction of tipping points, offering a more comprehensive approach. In addition, we propose a model for a more flexible ramping parameter in this framework, allowing us to potentially capture the behaviour of more complex systems.
    \item \textbf{In-depth derivations for a wide array of estimators:} For the estimators themselves we show detailed derivations and offer a couple of different options for some of the models. 
    \item \textbf{Implementations in \code{R}:} We provide implementations of extensively tested methods in the \code{R}-language. This includes examples of running the code as well as documentation of the overall structure of the code. Moreover, we implement an efficient estimator that to a great extent relies solely on numerical methods. 
    \item \textbf{Simulation study:} A detailed simulation study is conducted to evaluate the computational efficiency, error, robustness and numerical stability of these estimators. Furthermore, our simulation study constrasts for two of the models the more numerically founded estimator to our other estimators. 
    \item \textbf{Application to the AMOC:} One of the newly proposed models is together with the original model applied to a time series of a proxy for the strengh of the AMOC and the results are compared to the findings of previous work.
    \item \textbf{Future research:} We mention some of the interesting directions for further research that ur findings gives rise to.
\end{enumerate}
\subsection{Structure of the Thesis}
The thesis is structured in the following way: We start with a section about general methods, where we  introduce the part of the theory of stochastic differential equation that is necessary for the understanding of the tipping model and later do parameteric inference. Then we present the theory from dynamical systems to motivate the tipping model, which is eventually introduced. Hereafter we show the ideas and results we use to do parametric inference on this model. Finally, we explain how the models are implemented in \code{R} as well as how we can assess them. 

Then we have a simulation study showing 4 semi-independent experiments conducted on our methods. This all build up towards the practical application on the proxy of the AMOC strengh. We discuss the results from the simulation study and the tipping of the AMOC. In the end, we briefly conclude on the thesis and reflect on the methods in a more general sense. 