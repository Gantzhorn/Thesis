\subsection{Background and motivation}
Stochastic differential equations have long been a popular choice for modelling processes in physics, biology, and a wide range of other fields of study. An often celebrated quality of these models is their innate ability to capture the stochasticity of the dynamics in these systems. However, another situation in which they are frequently used is when one works with a system so complex that it is either not entirely understood or when the complexity makes the models intractable. In these cases, stochastic differential equations allow us, often with great success, to model the well understood- or simple parts as deterministic dynamics, whereas the unknown- or complex terms act as random. In this thesis, we exclusively work with one-dimensional stochastic differential equations driven by a standard Brownian motion; commonly written on the following parametric form
\begin{align}
    \mathrm{d}X_t = b(X_t, t;\theta)\mathrm{d}t + \sigma\left(X_t, t; \theta\right)\mathrm{d}W_t,
\end{align}
where the solution, $X_t\in\mathbb{R}$, is called an Ito process. The stochasticity in the system comes from, $W_t$, which is a Wiener process. The functions in the deterministic- and stochastic part are named drift- and diffusion terms, respectively, and our overall goal is typically to do some sort of inference about the parameter, $\theta\in\Theta\subseteq\mathbb{R}^p$, that the functions depend on. \\\\
An example of an application of stochastic differential equations is the modelling of bifurcation points, commonly known as tipping points in ecological systems. The study of bifucartion points stems from the study of deterministic dynamical systems. These systems are modelled by \textit{change} in the system depending on its current value. This could for instance be models where some value, $x_t\in\mathbb{R}$, is described by the differential equation
\begin{align}
    \mathrm{d}x_t = f(x_t, \lambda)\mathrm{dt},
\end{align}
where $f$ is some function that relates $x_t$ to its derivative and $\lambda$ is a so-called bifurcation parameter that the system also depends on. In the study of dynamical systems, the bifurcation parameter typically models changes in the overall structure of the dynamical system. In this regard, a bifurcation point is a value of the bifurcation parameter, $\lambda = \lambda_c$, where the system's qualitative structure changes significantly. As such, we model tipping points by considering systems that have this sort of qualitative change at some values of $\lambda$. we assume that the variable of interest, $x_t$, exists in one of a possible range of stable states and our goal is to construct a model that allows us to reason about when or what needs to be fulfilled for a \textit{transition} between two of those states to occur i.e. for $\lambda$ to somehow reach this bifurcation point or tipping point. In nature, there are countless systems that can be thought of as existing as two or more stable states, such as: The landmass of Greenland can exist in a state with or without an ice cap and the Amazon can exist in a state as a rainforest, a savannah or even as a desert. \\\\
Yet, this alone does not make it clear which type of dynamical system is appropriate as the overall dynamics of such systems are typically intricate. Even with extensive domain knowledge one is often at a loss, when trying to provide a comprehensive description of a system with complex phenomena such as tipping points. Luckily, tipping comes in different types and choosing a specific type of tipping for the model of our system can vastly simplify the dynamics locally. An example of this is the model used to estimate the tipping point of an important climate system: The Atlantic Meridional Overturning Current (AMOC) \cite{Ditlevsen2023}. The AMOC is a system of ocean currents and it plays a pivotal role in the Earth's climate. The system exists in two states: "on" and "off" referring to whether there is a current or not. The specific type of bifurcation this system can undergo is known as a saddle-node bifurcation \cite{Ditlevsen2023}\cite{Strogatz2019_gv}. As other bifurcations the saddle-node bifurcation has a so-called normal form of its dynamics 
\begin{align}
    \mathrm{d}x_t = \pm\left(x_t^2 - \lambda\right)\mathrm{d}t. \label{eq:normalFormIntroduction}
\end{align}
A normal form is an approximate form of the system's dynamics. Systems that can undergo saddle-node bifurcations are well described by this normal form, when the bifurcation parameter, $\lambda$, is close to the bifurcation point, $\lambda_c$, and this is exactly what is exploited in the model. We assume that we are sufficiently close to the tipping point such that the system's dynamics are well described by the normal form, then we model the tipping point by the following stochastic differential equation
\begin{align}
    \mathrm{d}X_t &= \pm(A(X_t - m)^2 + \lambda_t)\mathrm{d}t + \sigma\sqrt{\left(aX_t^2 + bX_t + c\right)}\mathrm{d}W_t, \label{eq:normalFormModelIntroduction}\\
    \lambda_t &= \lambda_0\left(1 - \max\left\{\frac{t - t_0}{\tau_c}\right\}\right)^\nu \label{eq:lambdaRampDefinitionIntroduction}
\end{align}
which is a system similar to the normal form (\ref{eq:normalFormIntroduction}) with the addition of a Wiener driven diffusion term and some parameters. Later, we argue how one easily arrives at (\ref{eq:normalFormModelIntroduction}) from (\ref{eq:normalFormIntroduction}). The parameters we want to estimate are $\theta = \left(A\; m\; \sigma\; \lambda_0\; \nu\right)^\top$. The values $a, b$ and $c$ are chosen, when one picks a specific diffusion such that the square root is well-defined. This way of modelling the bifurcation parameter, $\lambda_t$, assumes that we have observations from the system in some stable state before time $t_0$, after which a \textit{ramping} of the parameter begins. To say something about the tipping point, we naturally also need observations from after ramping has begun. We show that for this model the tipping point is $\lambda_c = 0$, according to the ramping model (\ref{eq:lambdaRampDefinitionIntroduction}) this corresponds to the point in time, $t_c = t_0 + \tau_c$.

The model we present is a generalization of \cite[equation (1)]{Ditlevsen2023} in two ways: Firstly, the noise in the original work was assumed additive. We extend this to include diffusion terms from the Pearson diffusions: A class of stochastic differential equations on the form
\begin{align}
    \mathrm{d}X_t &= -\alpha_0\left(X_t - \mu_0\right)\mathrm{d}t + \sigma\sqrt{\left(aX_t^2 + bX_t + c\right)}\mathrm{d}W_t. \label{eq:pearsonDiffusionsIntroduction}
\end{align}
The Pearson diffusions include a process with additive noise ― commonly known as the Ornstein-Uhlenbeck (OU) process. This process corresponds to picking $a = b = 0$ and $c = 1$ in (\ref{eq:pearsonDiffusionsIntroduction}). The understanding of the OU process is central to the formulation of the original stochastic saddle-node bifurcation model amongst other things, because they share diffusion terms. The OU process belongs to a special class of Pearson diffusions referred to as the \textit{ergodic} Pearson diffusions in the literature. This is a class of models for which there exists efficient estimators and as these stochastic differential equations share many properties with the Ornstein-Uhlenbeck process using their diffusion terms in the stochastic saddle-node bifurcation model is quite natural. In doing so, we introduce five novel versions of the stochastic normal form model for estimation of saddle-node bifurcations bringing the total number of available models up to six.

Secondly, the ramping in the original work \cite[equation (2)]{Ditlevsen2023} can be seen as a special case of (\ref{eq:lambdaRampDefinitionIntroduction}) with $\nu = 1$. We shall see that changing $\nu$ can alter the dynamics quite drastically. However, the extension is nevertheless natural in that it preserves the way we think about the tipping point in the model as a tipping time. To this end, this point in time is still given by $t_c = t_0 + \tau_c$, according to the model.\\\\
Our work with these models focuses on and solves some of the challenges that come with both types of extensions. This includes the use of the Strang splitting scheme\cite{SplittingSchemes} to approximate the transition density of (\ref{eq:normalFormModelIntroduction}) as well as the use of the rich framework for numerical optimization provided by the \code{R}-language along with our own implementations of sophisticated numerical methods. \cite{Rlang} In practical implementations, the general aim is, of course, to keep the methods as efficient and numerically stable as possible. Although, our work also focuses on the consequences of varying degrees of automation in the implementations. Specifically, we examine the effect that relying more on numerical methods over analytical results has on the quality of our estimator, the computational speed and numerical stability. To this end, we also consider how easy it is to use estimators based on these different degrees of automation in practical applications. Finally, with one of our newly developed models, we investigate the estimated tipping time of the AMOC and compare it with the estimates found in the original study.
\subsection{Key contributions}
For a quick overview, we list the key contributions of this thesis.
\begin{enumerate}
    \item \textbf{Model development and -extensions:} We introduce five models that take the diffusion terms from the Pearson diffusions and use them to extend the existing methodology for the prediction of tipping points, offering a more comprehensive approach. In addition, we propose a model for a more flexible ramping parameter in this framework, allowing us to potentially capture the behaviour of other systems.
    \item \textbf{In-depth derivations for a wide array of estimators:} For the estimators themselves we show detailed derivations and offer a couple of different options for some of the models. 
    \item \textbf{Implementations in \code{R}:} We provide implementations of extensively tested methods in the \code{R} language. This includes examples of running the code as well as documentation of the overall structure of the code. Moreover, we implement an efficient estimator that to a great extent relies solely on numerical methods. 
    \item \textbf{Simulation study:} A detailed simulation study is conducted to evaluate the computational efficiency, error, robustness and numerical stability of these estimators. Furthermore, our simulation study contrasts for two of the models the more numerically founded estimator to our other estimators. 
    \item \textbf{Application to the AMOC:} One of the newly proposed models is together with the original model applied to a time series of a proxy for the strengh of the AMOC and the results are compared to the findings of previous work.
    \item \textbf{Future research:} We mention some interesting directions for further research that our findings give rise to.
\end{enumerate}
\subsection{Structure of the Thesis}
The thesis is structured in the following way: We start with a section about our general methods, where we  introduce the part of the theory of stochastic differential equation that is necessary for the understanding of the tipping model and to later be able to do parametric inference. Then we present the theory from dynamical systems that motivates and allows us to understand the tipping model, which is eventually introduced. Hereafter we show the central ideas and results, we use to do parametric inference in the tipping model. Finally, we explain how methods for the estimators are implemented in \code{R} as well as how we assess them. \\\\
Then we have a simulation study showing 4 semi-independent experiments conducted on our methods. Not only does our simulation study show interesting results in themselves, but they also naturally build up towards the practical application on the proxy of the AMOC strength. We discuss the results from the simulation study and the tipping of the AMOC. In the end, we briefly conclude on the thesis and reflect on the methods in a more general sense. 